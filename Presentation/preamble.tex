% There are many different themes available for Beamer. A comprehensive
% list with examples is given here:
% http://deic.uab.es/~iblanes/beamer_gallery/index_by_theme.html
% You can uncomment the themes below if you would like to use a different
% one:
%\usetheme{AnnArbor}
%\usetheme{Antibes}
%\usetheme{Bergen}
%\usetheme{Berkeley}
%\usetheme{Berlin}
%\usetheme{Boadilla}
%\usetheme{boxes}
%\usetheme{CambridgeUS}
%\usetheme{Copenhagen}
%\usetheme{Darmstadt}
%\usetheme{default}
%\usetheme{Frankfurt}
%\usetheme{Goettingen}
%\usetheme{Hannover}
%\usetheme{Ilmenau}
%\usetheme{JuanLesPins}
%\usetheme{Luebeck}
\usetheme{Madrid}
%\usetheme{Malmoe}
%\usetheme{Marburg}
%\usetheme{Montpellier}
%\usetheme{PaloAlto}
%\usetheme{Pittsburgh}
%\usetheme{Rochester}
%\usetheme{Singapore}
%\usetheme{Szeged}
%\usetheme{Warsaw}

\usepackage{graphicx} % inclusion des figures
\usepackage{multimedia} %inclusion de vidéos
\usepackage{url}

\usepackage{amsmath} % collection de symboles mathématiques
\usepackage{amssymb} % collection de symboles mathématiques
\everymath{\displaystyle} % affichage en mode bloc pour toutes les équations
\usepackage{epstopdf} % Permet la conversion des .eps en .pdf pour les inclure
\usepackage[utf8]{inputenc}       % utilisation directe des caractÚres accentués sur pc
\usepackage[T1]{fontenc} % codage moderne des caractÚres sous Latex
\usepackage{upgreek} % lettres grecques
\usepackage{nameref} % pour désigner des parties par leur nom
\usepackage{url} % pour mettre des URL
\usepackage{pgf, tikz} % tikz pour dessiner
\usepackage{float}
\usepackage[french]{babel}
\usepackage{subcaption}

\defbeamertemplate*{footline}{mytheme}%
{  \begin{beamercolorbox}[ht=2.5ex]{bordure}
\begin{beamercolorbox}[wd=0.2\paperwidth,ht=2.5ex,dp=1ex,center]{author in head/foot}%
\usebeamerfont{author in head}\insertshortauthor
\end{beamercolorbox}%
% \begin{beamercolorbox}[wd=0.34\paperwidth,ht=2.5ex,dp=1ex,center]{title in head/foot}%
% \usebeamerfont{title in head/foot}\insertshorttitle
% \end{beamercolorbox}%
% \begin{beamercolorbox}[wd=0.4\paperwidth,ht=2.5ex,dp=1ex,center]{title in head/foot}%
% \usebeamerfont{title in head/foot}\insertdate
% \end{beamercolorbox}%
\begin{beamercolorbox}[wd=0.74\paperwidth,ht=2.5ex,dp=1ex,center]{title in head/foot}%
\usebeamerfont{title in head/foot}\insertdate
\end{beamercolorbox}%
\begin{beamercolorbox}[wd=0.06\paperwidth,ht=2.5ex,dp=1ex,right]{date in head/foot}%
\usebeamerfont{date in head/foot}\insertframenumber /\inserttotalframenumber{}\hspace{2ex}
\end{beamercolorbox}%

  \end{beamercolorbox}
}

\defbeamertemplate*{headline}{mytheme}%
{  \begin{beamercolorbox}[ht=2.5ex]{bordure}
\begin{beamercolorbox}[wd=\paperwidth,ht=2.5ex,dp=2ex]{author in head/foot}%
\usebeamerfont{author in head}\vskip6pt\insertsectionnumber.~\insertsection \hfill \insertsubsection
\end{beamercolorbox}%

  \end{beamercolorbox}
}

\providecommand\encircle[1]{%
  \tikz[baseline=(X.base)] 
    \node (X) [draw, shape=circle, inner sep=0] {\strut #1};}

\providecommand{\appropto}{\mathrel{\vcenter{
  \offinterlineskip\halign{\hfil$##$\cr
    \propto\cr\noalign{\kern2pt}\sim\cr\noalign{\kern-2pt}}}}}
    
%%\AtBeginSubsection[]
%%{
%%  \begin{frame}<beamer>{}
%%    \tableofcontents[currentsection,currentsubsection]
%%  \end{frame}
%%}
\AtBeginSection[]
{
  \begin{frame}<beamer>{}
    \tableofcontents[currentsection]
  \end{frame}
}

\providecommand{\so}{\mathfrak{so}}
\providecommand{\drawat}[3]{\makebox[0pt][l]{\raisebox{#2}{\hspace*{#1}#3}}}
\providecommand{\p}{\partial}
\providecommand{\ds}{\displaystyle}
\providecommand{\vs}{\vspace{0.3cm}}
\providecommand{\e}{\mathbf{e}}
\providecommand\fd[1]{\mathbf{#1}}
\providecommand\ffd[1]{\boldsymbol{#1}}
\providecommand\demi{\frac{1}{2}}
\providecommand\indi{\hat{\imath}}
\providecommand\indj{\hat{\jmath}}
\providecommand\inda{\hat{\mu}}
\providecommand\indb{\hat{\nu}}
\providecommand\indl{\hat{l}}
\providecommand\indk{\hat{k}}
\providecommand\xii{x_{\indi}}
\providecommand\xjj{x_{\indj}}
\providecommand\indc{\hat{\rho}}
\providecommand\indd{\hat{\beta}}
\providecommand{\grand}[1]{[#1,\cdot]}
\providecommand{\petit}[1]{[\cdot ,#1]}
\providecommand{\seti}{\llbracket 1,n \rrbracket}
\providecommand{\Qx}{Q\backslash \{x\}}
\providecommand{\TQ}{\mathcal{T}_Q}
\providecommand{\TQp}{\mathcal{T}_{Q'}}

\newtheorem{theo}{Théorème}
\newtheorem{defi}{Définition}
\setbeamertemplate{blocks}[rounded][shadow=true]

\providecommand{\isEquivTo}[1]{\underset{#1}{\sim}}

\providecommand{\appropto}{\mathrel{\vcenter{
  \offinterlineskip\halign{\hfil$##$\cr
    \propto\cr\noalign{\kern2pt}\sim\cr\noalign{\kern-2pt}}}}}
