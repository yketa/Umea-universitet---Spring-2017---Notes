\documentclass[class=report, float=false, crop=false]{standalone}
%  \usepackage[subpreambles=true]{standalone}

\usepackage{pgf, tikz}
\usetikzlibrary{shapes.misc}
\usetikzlibrary{decorations.pathreplacing}

\tikzset{cross/.style={cross out, draw=black, minimum size=2*(#1-\pgflinewidth), inner sep=0pt, outer sep=0pt},
%default radius will be 1pt. 
cross/.default={0.25pt},
    point/.style={
    thick,
    draw=black,
    cross out,
    inner sep=0pt,
    minimum width=4pt,
    minimum height=4pt,
    },
}

\graphicspath{{introduction}}

% \begin{cbunit}

\begin{document}

\chapter*{Introduction}
\label{introduction}
\addcontentsline{toc}{chapter}{Introduction}

Granular materials are ubiquitous in nature and are the second-most manipulated material in industry \cite{patrick2005slow}. These are non-thermal systems, \textit{i.e.} their kinetic energy vanishes in absence of an external drive. Despite their relative simplicity, they exhibit a wide range of interesting -- even fascinating -- behaviours \cite{youtube4}.\\

Granular matter can show either liquid-like or solid-like properties under different experimental conditions. This transition from a flowing state to a rigid state is known as the jamming transition. Understanding why and how these materials transition from one of this state to the other is a great challenge of our times.\\

**** granular materials have lots of interesting and intriguing properties, yet too little is known about them, their understanding is great challenge of our time and would have lots of theoretical (phase transitions) and practical (natural catastrophes) implications \cite{youtube4} ****\\

**** modifications which have been added to currently existing C algorithms ****\\

**** simulations with 64 particles on 1 core and 1024 particles on 4 cores on computers provided by the Department of Physics at Umeå University, simulations with 16384 particles on 12 cores on super computer provided by the Swedish National Infrastructure for Computing (SNIC) at High Performance Computing Center North (HPC2N) ****

% \addcontentsline{toc}{section}{References}
\bibliographystyle{unsrtnat}
\bibliography{references/biblio}
{\renewcommand{\bibname}{References}\bibliography{references/biblio}}

\end{document}

% \end{cbunit}