\documentclass[class=report, float=false, crop=false]{standalone}
%  \usepackage[subpreambles=true]{standalone}

\usepackage{pgf, tikz}
\usetikzlibrary{shapes.misc}
\usetikzlibrary{decorations.pathreplacing}

\tikzset{cross/.style={cross out, draw=black, minimum size=2*(#1-\pgflinewidth), inner sep=0pt, outer sep=0pt},
%default radius will be 1pt. 
cross/.default={0.25pt},
    point/.style={
    thick,
    draw=black,
    cross out,
    inner sep=0pt,
    minimum width=4pt,
    minimum height=4pt,
    },
}

\graphicspath{{introduction}}

% \begin{cbunit}

\begin{document}

\chapter*{Introduction}
\label{introduction}
\addcontentsline{toc}{chapter}{Introduction}

Granular materials are ubiquitous in nature and are the second-most manipulated material in industry \cite{patrick2005slow}. These are non-thermal systems, \textit{i.e.} their kinetic energy vanishes in absence of an external drive. Despite their relative simplicity, they exhibit a wide range of interesting -- even fascinating -- behaviours \cite{youtube4}.\\

Granular matter can show either liquid-like or solid-like properties under different experimental conditions. This transition from a flowing state to a rigid state is known as the \textit{jamming transition}, and has been the subject of much recent work. Understanding why and how these materials transition from one of this state to the other is indeed of great theoretical interest, \textit{e.g.} understanding the mechanisms which govern jamming in athermal macroscopic systems may help us to understand how supercooled liquid freeze into a frozen glass \cite{liu1998nonlinear}.\\

Numerical shearing simulations of non-rotating frictionless soft-core disks have been proven particularly efficient to study the jamming transition \cite{PRL99.178001,PRE83.031307}. Our present report aims to present the methods which have been employed in such studies and to show how they can and have been modified and enriched to study more general rotating frictionless soft-core spheroids (\textit{ie.} ellipsoids of revolution), both prolate -- elongated -- and oblate -- flattened.\\

Computer programs were written in MATLAB language for evaluation of the methods then in C language for performance (speed of execution). All the code developed specifically for this project is available on GitHub, in the repository \href{https://github.com/yketa/Umea-universitet---Spring-2017---code}{yketa/Umea-universitet-{}-{}-Spring-2017-{}-{}-code}. Simulations with 64 and 1024 particles were run on 1 and 4 cores respectively on computers provided by the Department of Physics at Umeå University. Simulations with 16384 particles were run on 12 cores on the supercomputer provided by the Swedish National Infrastructure for Computing (SNIC) at High Performance Computing Center North (HPC2N).\\

This report is organised as follows. In \hyperref[chap:quaternions]{chapter \ref{chap:quaternions}}, we show how quaternions are used to numerically integrate rigid body rotation equations. In \hyperref[chap:ellipsoids]{chapter \ref{chap:ellipsoids}}, we briefly introduce ellipsoids and present different numerical methods to determine if and how they overlap. In \hyperref[chap:jamming]{chapter \ref{chap:jamming}}, we more thoroughly describe the jamming transition, as well as the models and methods we have used, modified and improved. In \hyperref[chap:results]{chapter \ref{chap:results}} we present and discuss our numerical results.\\

Appendices provide theoretical backgrounds on objects and theories we use throughout this report. \hyperref[appendix:quaternions]{Appendix \ref{appendix:quaternions}} is a theoretical introduction to quaternions. \hyperref[appendix:euler]{Appendix \ref{appendix:euler}} is a demonstration of the Euler's equation describing the rotation of a rigid body. \hyperref[appendix:ellipsoids]{Appendix \ref{appendix:ellipsoids}} is a theoretical introduction to ellipsoids. \hyperref[appendix:rg]{Appendix \ref{appendix:rg}} is an introduction to renormalisation group theory and aims to show how this theory allows us to make scaling assumptions.

% \addcontentsline{toc}{section}{References}
\bibliographystyle{unsrtnat}
\bibliography{references/biblio}
{\renewcommand{\bibname}{References}\bibliography{references/biblio}}

\end{document}

% \end{cbunit}