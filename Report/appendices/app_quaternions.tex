\documentclass[class=report, float=false, crop=false]{standalone}
\usepackage[subpreambles=true]{standalone}

\usepackage{pgf, tikz}
\usetikzlibrary{shapes.misc}
\usetikzlibrary{decorations.pathreplacing}

\tikzset{cross/.style={cross out, draw=black, minimum size=2*(#1-\pgflinewidth), inner sep=0pt, outer sep=0pt},
%default radius will be 1pt. 
cross/.default={0.25pt},
    point/.style={
    thick,
    draw=black,
    cross out,
    inner sep=0pt,
    minimum width=4pt,
    minimum height=4pt,
    },
}

\graphicspath{{figures/images/}}

% \begin{cbunit}

\begin{document}

\chapter{Quaternions}
\label{appendix:quaternions}

\section{Definition}

Quaternions were introduced by Hamilton as extension of complex numbers \cite{shoemake}:
\begin{equation}
q = q_0 + \sum_{i=1}^{3}q_i e_i\text{, with } \begin{cases} e_m^2 = -1& \\ e_me_n = \varepsilon_{mnk} e_k&\text{, } m \neq n \end{cases} \text{ and } q_0,q_1,q_2,q_3 \in \mathbb{R}
\end{equation}

For convenience, we will write $q \equiv [\vec{q},q_0]$ with $\vec{q} = (q_1,q_2,q_3) \in \mathbb{R}^3$, and $\mathbb{H}$ the quaternions' space.\\

Moreover, we will identify vectors $\vec{v}$ from $\mathbb{R}^3$ with $[\vec{v},0]$ and scalars $w$ from $\mathbb{R}$ with $[0,w]$. We will assume this identification throughout the document without notification.

\section{Properties}
\label{quat_properties}

\begin{description}
\item[Addition] $\boxed{q + q' = [\vec{q} + \vec{q'},q_0 + q'_0]}$
\item[Multiplication] $\boxed{qq' = [\vec{q}\times\vec{q'} + q_0\vec{q'} + q'_0\vec{q},q_0q'_0 - \vec{q}\cdot\vec{q'}]}$
\begin{align*}
qq' &= (q_0 + \sum_{i=1}^{3}q_i e_i)(q'_0 + \sum_{i=1}^{3}q'_i e_i) = q_0q'_0 + \underbrace{q_0\sum_{i=1}^{3}q'_i e_i}_{q_0\vec{q'}} + \underbrace{q'_0\sum_{i=1}^{3}q_i e_i}_{q'_0\vec{q}} + \underbrace{\sum_{i=1}^{3}q_iq'_i \underbrace{e_ie_i}_{-1}}_{-\vec{q}\cdot\vec{q'}} + \underbrace{\sum_{\substack{i,j \\ i \neq j}}q_iq'_j\underbrace{e_ie_j}_{\varepsilon_{ijk}e_k}}_{\vec{q}\times\vec{q'}}\\
&= [\vec{q}\times\vec{q'} + q_0\vec{q'} + q'_0\vec{q},q_0q'_0 - \vec{q}\cdot\vec{q'}]
\end{align*}
\item[Multiplication by a scalar] $\boxed{\lambda q = [\lambda\vec{q},\lambda q_0]}$
\item[Bilinearity] $\boxed{q''(\lambda q + \lambda q') = \lambda q''q + \lambda q''q'}$
\item[Associative property] $\boxed{(qq')q'' = q(q'q'')}$
\item[Conjugate] $\boxed{q^* = [-\vec{q},q_0]}$
\begin{align*}
q^* &= (q_0 + \sum_{i=1}^{3}q_i e_i)^* = q_0 - \sum_{i=1}^{3}q_i e_i\\
&= [-\vec{q},q_0]
\end{align*}
\item[Conjugation property] $\boxed{(qq')^* = q'^*q^*}$
\begin{align*}
(qq')^* &= ([\vec{q},q_0][\vec{q'},q'_0])^*\\
&= ([\vec{q}\times\vec{q'} + q_0\vec{q'} + q'_0\vec{q},q_0q'_0 - \vec{q}\cdot\vec{q'}])^*\\
&= [\underbrace{-\vec{q}\times\vec{q'}}_{(-1)\vec{q'}\times(-1)\vec{q}} + q_0(-\vec{q'}) + q'_0(-\vec{q}),q_0q_0' - (-1)\vec{q}\cdot(-1)\vec{q'}]\\
&= [-\vec{q'},q'_0][-\vec{q},q_0]\\
&= q'^*q^*
\end{align*}
\item[Norm] $\boxed{N^2(q) = \sum_{i=0}^{3} q_i^2}$
\begin{align*}
N^2(q) &= qq* = [\vec{q},q_0][-\vec{q},q_0] = [\cancel{-q_0 \vec{q}} + \cancel{q_0\vec{q}},q_0q_0 + \vec{q}\cdot\vec{q}]\\
&= [0,\sum_{i=0}^{3} q_i^2] = \sum_{i=0}^{3} q_i^2
\end{align*}
\item[Inverse] $\boxed{q^{-1} = q^*/N^2(q)}$
\begin{align*}
q\frac{q*}{N^2(q)} = \frac{q*}{N^2(q)}q = 1 \Leftrightarrow q^{-1} = \frac{q*}{N^2(q)}
\end{align*}
\item[Action of an unit quaternion] From the property of the inverse of a quaternion, it is trivial that the action of any quaternion $q$
\begin{align*}
\mathcal{A}_q \colon &\mathbb{H} \to \mathbb{H}\\     &\phantomarrow{\mathbb{H}}{p} qpq^{-1}
\end{align*}
will be the same as the action of the associated unit quaternion $q/N(q)$. Therefore, we will only consider unit quaternion from now on and use $q^*$ instead of $q^{-1}$.\\

We then have that for any $p = [\vec{v},w]$, $\boxed{qpq^* = [\vec{v'},w]}$ with $N(\vec{v}) = N(\vec{v'})$
\begin{align*}
&\text{Indeed, with $S(q) \equiv \frac{q+q^*}{2}$ the scalar part of any quaternion $q$ we have}\\
%\vspace{-30pt}
&2S(qpq^*)\\
&= qpq^* + (qpq^*)^* = qpq^* + qp*q^*\\
&= q\underbrace{(p + p^*)}_{2w}q^* = 2w\cancel{qq^*} = 2w\\
&= 2S(p)\\
&\text{The conjugation property ensures that the multiplication conserves the}\\
&\text{norm, then $N(qpq^*)=N(p)$, hence the property.}
\end{align*}
Therefore the action of any quaternion on any vector of $\mathbb{R}^3$ is a vector of $\mathbb{R}^3$.
\end{description}

\section{Actions and rotations}
\label{action_rotation}

Consider $\vec{v_0},\vec{v_1} \in \mathbb{R}^3$ with $\vec{v_0} \times \vec{v_1} \neq 0$ and $N(\vec{v_0}) = N(\vec{v_1}) = 1$. We define $q \equiv \vec{v_1}\vec{v_0}^* = [\vec{v_0}\times\vec{v_1},\vec{v_0}\cdot\vec{v_1}]$, $\theta \equiv (\vec{v_0};\vec{v_1})$ and $\vec{v} = \frac{\vec{v_0} \times \vec{v_1}}{||\vec{v_0} \times \vec{v_1}||}$ so that $q = [\vec{v}\sin\theta,\cos\theta]$. We want to know what is the action of $q$, $\mathcal{A}_q$, on any vector of $\mathbb{R}^3$.\\

We have $(\vec{v_0},\vec{v_1},\vec{v})$  a basis of $\mathbb{R}^3$. Thanks to the bilinearity of the product of quaternions, we can infer $\mathcal{A}_q(\vec{u}) \forall \vec{u} \in \mathbb{R}^3$ from the action of $q$ on this basis. The following demonstration was originally made in \cite{shoemake}.\\

\begin{description}
\item[Action on $\vec{v}$\hspace{20pt}]
\begin{align*}
\mathcal{A}_q(\vec{v}) &= q\vec{v}q^*\\
&= [\vec{v}\sin\theta,\cos\theta][\vec{v},0][-\vec{v}\sin\theta,\cos\theta]\\
&= [\vec{v}\cos\theta,-\underbrace{|\vec{v}|^2}_{=1}\sin\theta][-\vec{v}\sin\theta,\cos\theta]\\
&= [\underbrace{\vec{v}\sin^2\theta + \vec{v}\cos^2\theta}_{=\vec{v}},\cancel{-\sin\theta\cos\theta} + \underbrace{|\vec{v}|^2}_{=1}\cancel{\sin\theta\cos\theta}]\\
&= [\vec{v},0]\\
&= \vec{v}
\end{align*}
\item[Action on $\vec{v_0}$\hspace{20pt}] We define $\vec{v_2} \equiv \mathcal{A}_q(\vec{v_0}) = q\vec{v_0}q^*$ and notice that
\begin{align*}
\vec{v_2}\vec{v_1}^* &= (q\vec{v_0}q^*)\vec{v_1}^*\\
&= (q\vec{v_0}(\vec{v_1}\vec{v_0}^*)^*)\vec{v_1}^*\\
&= q\vec{v_0}(\vec{v_0}\vec{v_1}^*)\vec{v_1}^*\\
&=q\underbrace{(\vec{v_0}\vec{v_0})}_{=-1}\underbrace{(\vec{v_1}^*\vec{v_1}^*)}_{=-1}\\
&=q=\vec{v_1}\vec{v_0}^*
\end{align*}
so \(\begin{cases} \vec{v_1}\times\vec{v_2} &= \vec{v_0}\times\vec{v_1} \\ \vec{v_1}\cdot\vec{v_2} &= \vec{v_0} \cdot \vec{v_1} \end{cases} \). Therefore, $\vec{v_0}, \vec{v_1}$ and $\vec{v_2}$ belong to the same plane and furthermore, $(\vec{v_1};\vec{v_2}) = (\vec{v_0};\vec{v_1}) = \theta$. Then, $\vec{v_2}$ is the rotation of $\vec{v_0}$ of an angle $2\theta$ around $\vec{v}$.\\
\item[Action on $\vec{v_1}$\hspace{20pt}] We define $\vec{v_3} \equiv \mathcal{A}_q(\vec{v_1}) = q\vec{v_1}q^*$ and notice that
\begin{align*}
\vec{v_3}\vec{v_2}^* &= (q\vec{v_1}q^*)(q\vec{v_0}q^*)^*\\
&= (q(q\vec{v_0})q^*)(q\vec{v_0}q^*)^*\\
&=  q(q\vec{v_0}q^*)(q\vec{v_0}q^*)^*\\
&=q=\vec{v_1}\vec{v_0}^*
\end{align*}
so \(\begin{cases} \vec{v_2}\times\vec{v_3} &= \vec{v_0}\times\vec{v_1} \\ \vec{v_3}\cdot\vec{v_3} &= \vec{v_0} \cdot \vec{v_1} \end{cases} \). Therefore, $\vec{v_0}, \vec{v_1}, \vec{v_2}$ and $\vec{v_3}$ belong to the same plane and furthermore, $(\vec{v_2};\vec{v_3}) = (\vec{v_0};\vec{v_1}) = \theta$. Then, $\vec{v_3}$ is the rotation of $\vec{v_1}$ of an angle $2\theta$ around $\vec{v}$.\\
\end{description}

Finally, we have that $\mathcal{A}_q$ corresponds to the rotation of angle $2\theta$ around $\vec{v}$. The corollary to that property is that

\begin{center}$\boxed{\text{\textbf{every 3D rotation can be expressed as the action of some unit quaternion}}}$.\end{center}

% \addcontentsline{toc}{section}{References}
\bibliographystyle{unsrtnat}
\bibliography{references/biblio}
{\renewcommand{\bibname}{References}\bibliography{references/biblio}}

\end{document}

% \end{cbunit}