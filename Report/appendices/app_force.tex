\documentclass[class=report, float=false, crop=false]{standalone}
\usepackage[subpreambles=true]{standalone}

\usepackage{pgf, tikz}
\usetikzlibrary{shapes.misc}
\usetikzlibrary{decorations.pathreplacing}

\tikzset{cross/.style={cross out, draw=black, minimum size=2*(#1-\pgflinewidth), inner sep=0pt, outer sep=0pt},
%default radius will be 1pt. 
cross/.default={0.25pt},
    point/.style={
    thick,
    draw=black,
    cross out,
    inner sep=0pt,
    minimum width=4pt,
    minimum height=4pt,
    },
}

\graphicspath{{figures/images/}}

% \begin{cbunit}

\begin{document}

\chapter{Elastic force between ellipsoids}
\label{appendix:force}

\section{Reduced belonging matrix}

An ellipsoid $\mathcal{A}$ of centre $\vec{v}$ and semi-axes $(R_i)_{i=1:3}$ can be described by the matrix
\begin{equation}
\bar{B}(O,(R_i)_{i=1:3}) \equiv O \text{diag}(R_i^{-2})_{i=1:3} O^T
\end{equation}
where $O$ is the symmetric rotation matrix corresponding to the change of basis from the ellipsoid frame to the reference frame, such as
\begin{equation}
\forall \vec{r} \in \mathbb{R}^3, \begin{cases} (\vec{r} - \vec{v})^T \bar{B} (\vec{r} - \vec{v}) < 1 &\text{ if } \vec{r} \in \mathcal{A} \setminus \bar{\mathcal{A}} \\ (\vec{r} - \vec{v})^T \bar{B} (\vec{r} - \vec{v}) = 1 &\text{ if } \vec{r} \in \bar{\mathcal{A}} \\ (\vec{r} - \vec{v})^T \bar{B} (\vec{r} - \vec{v}) > 1 &\text{ if } \vec{r} \notin \mathcal{A} \end{cases}
\end{equation}
and with which we can write
\begin{equation}
\mu^2(\vec{r}\in\mathbb{R}^3) = (\vec{r}-\vec{v})^T\bar{B}(O,(R_i)_{i=1:3})(\vec{r}-\vec{v})
\label{rescaling_factor_squared}
\end{equation}
where $\mu(\vec{r})$ is the rescaling factor that has to be applied to the axes of $\mathcal{A}$ for $\vec{r}$ to be on its surface.\\

For simplicity, we will write $\bar{B}_{\mathcal{A}} \equiv \bar{B}(O,(R_i)_{i=1:3})$.\\

We have for all $\vec{r}$ on the surface of $\mathcal{A}$ that $\mu^2(\vec{r})=1$, therefore the surface of $\mathcal{A}$ is an isosurface of $\mu^2$. Therefore, we can conclude that the vector
\begin{equation}
\vec{\nabla}\mu^2(\vec{r}) = 2\bar{B}_{\mathcal{A}}(\vec{r}-\vec{v})
\end{equation}
is orthogonal to the surface of $\mathcal{A}$ in $\vec{r}$.

\section{Force}

We assumed that the force exerted on an ellipsoid $\mathcal{A}$ by an ellipsoid $\mathcal{B}$ of centre $\vec{r}_{\mathcal{A}}$ and $\vec{r}_{\mathcal{B}}$ respectively is
\begin{equation}
\vec{f}_{\mathcal{A}\mathcal{B}}^{\text{el}}(\vec{r_{\mathcal{A}}},\vec{r_{\mathcal{B}}}) = k_e (1 - \mu(\vec{r_{\mathcal{A}}},\vec{r_{\mathcal{B}}}))\frac{d\mu}{d\vec{r_{\mathcal{A}}}}(\vec{r_{\mathcal{A}}},\vec{r_{\mathcal{B}}})
\end{equation}
according to equation \ref{pre_force_el_ellipsoids}, where $\mu(\vec{r}_{\mathcal{A}},\vec{r}_{\mathcal{B}})$ is the rescaling factor that has to be applied to both $\mathcal{A}$ and $\mathcal{B}$ for them to be externally tangent.\\

Contrarily to the case of spheres, we have for ellipsoids that the rescaling factor $\mu(\vec{r_{\mathcal{A}}},\vec{r_{\mathcal{B}}})$ is not uniquely determined by the positions of their centres. Indeed, if we were to add $\vec{dr_{\mathcal{A}}}$ to the position $\vec{r_{\mathcal{A}}}$ of the centre of ellipsoid $\mathcal{A}$, the contact point $\vec{r_C}$ with ellipsoid $\mathcal{B}$ would move as well.\\

Since we are looking for the first derivative of the quantity $\mu(\vec{r_{\mathcal{A}}},\vec{r_{\mathcal{B}}})$, we will approximate rescaled ellipsoids by their respective tangent planes at $\vec{r_C}$ as suggested by \cite{donev}.\\

\begin{figure}[h!]
\centering
\includestandalone{figures/tikz/derivative_rescaling_factor}
\caption{Ellipsoids rescaled with rescaling factor $\mu(\vec{r_{\mathcal{A}}},\vec{r_{\mathcal{B}}})$ and their contact plane in plain thick trait. Ellipsoids rescaled with rescaling factor $\mu(\vec{r_{\mathcal{A}}} + \vec{dr_{\mathcal{A}}},\vec{r_{\mathcal{B}}})$ and their contact plane in dashed trait.}
\label{contact_points_fig}
\end{figure}

If ellipsoid $\mathcal{A}$ is moved by $\vec{dr_{\mathcal{A}}}$, there appears a gap $dh$ between the tangent planes of the rescaled ellipsoids with
\begin{equation}
dh = \vec{dr_{\mathcal{A}}}\cdot\vec{n}_{\mathcal{A}}(\vec{r_C})
\label{dh}
\end{equation}
where $\vec{n}_{\mathcal{A}}(\vec{r_C})$ is the outward-facing unitary surface vector of ellipsoid $\mathcal{A}$ in $\vec{r_C}$.\\

To close this gap, we have to rescale the -- yet rescaled -- ellipsoids with a factor $\nu$ so that
\begin{align*}
% &d\nu~(\vec{r_C} - \vec{r_{\mathcal{A}}})\cdot\vec{n}_{\mathcal{A}}(\vec{r_C}) - d\nu~(\vec{r_C} - \vec{r_{\mathcal{B}}})\cdot\vec{n}_{\mathcal{A}}(\vec{r_C}) = dh\\
% \Leftrightarrow~ &d\nu~ (\vec{r_{\mathcal{B}}} - \vec{r_{\mathcal{A}}})\cdot\vec{n}_{\mathcal{A}}(\vec{r_C}) = dh\\
% \Leftrightarrow~ &d\nu = \frac{dh}{(\vec{r_{\mathcal{B}}} - \vec{r_{\mathcal{A}}})\cdot\vec{n}_{\mathcal{A}}(\vec{r_C})}
\left((\vec{r_C} - \vec{r_{\mathcal{A}}}) - \nu (\vec{r_C} - \vec{r_{\mathcal{A}}})\right)\cdot\vec{n}_{\mathcal{A}}(\vec{r_C}) - \left((\vec{r_C} - \vec{r_{\mathcal{B}}}) - \nu (\vec{r_C} - \vec{r_{\mathcal{B}}})\right)\cdot\vec{n}_{\mathcal{A}}(\vec{r_C}) = dh \Rightarrow 1 - \nu = \frac{dh}{(\vec{r_{\mathcal{B}}} - \vec{r_{\mathcal{A}}})\cdot\vec{n}_{\mathcal{A}}(\vec{r_C})}
\end{align*}
and
\begin{align*}
\mu(\vec{r_{\mathcal{A}}} + \vec{dr_{\mathcal{A}}},\vec{r_{\mathcal{B}}}) = \nu\mu(\vec{r_{\mathcal{A}}},\vec{r_{\mathcal{B}}}) \Rightarrow \vec{dr_{\mathcal{A}}}\cdot\frac{d\mu}{\vec{dr_{\mathcal{A}}}}(\vec{r_{\mathcal{A}}},\vec{r_{\mathcal{B}}}) = -\mu(\vec{r_{\mathcal{A}}},\vec{r_{\mathcal{B}}})(1 - \nu)
\end{align*}
which, with equation \ref{dh}, leads to
\begin{align*}
\frac{d\mu}{d\vec{r_{\mathcal{A}}}}(\vec{r_{\mathcal{A}}},\vec{r_{\mathcal{B}}}) = -\mu(\vec{r_{\mathcal{A}}},\vec{r_{\mathcal{B}}})\frac{\vec{n}_{\mathcal{A}}(\vec{r_C})}{(\vec{r_{\mathcal{B}}} - \vec{r_{\mathcal{A}}})\cdot\vec{n}_{\mathcal{A}}(\vec{r_C})}
\end{align*}
We can then conclude, with equation \ref{surface_vec_reduced}, that
\begin{equation}
\frac{d\mu}{d\vec{r_{\mathcal{A}}}}(\vec{r_{\mathcal{A}}},\vec{r_{\mathcal{B}}}) = -\frac{\mu(\vec{r_{\mathcal{A}}},\vec{r_{\mathcal{B}}})}{(\vec{r_{\mathcal{B}}} - \vec{r_{\mathcal{A}}})^T\bar{B}_{\mathcal{A}}(\vec{r_C} - \vec{r_{\mathcal{A}}})}\bar{B}_{\mathcal{A}}(\vec{r_C} - \vec{r_{\mathcal{A}}})
\end{equation}
Therefore, with equation \ref{pre_force_el_ellipsoids}, we finally have that
\begin{equation}
\boxed{\vec{f}_{\mathcal{A}\mathcal{B}}^{\text{el}}(\vec{r_{\mathcal{A}}},\vec{r_{\mathcal{B}}}) = -k_e \frac{\mu(\vec{r_{\mathcal{A}}},\vec{r_{\mathcal{B}}})\left(1 - \mu(\vec{r_{\mathcal{A}}},\vec{r_{\mathcal{B}}})\right)}{(\vec{r_{\mathcal{B}}} - \vec{r_{\mathcal{A}}})^T\bar{B}_{\mathcal{A}}(\vec{r_C} - \vec{r_{\mathcal{A}}})}\bar{B}_{\mathcal{A}}(\vec{r_C} - \vec{r_{\mathcal{A}}})}
\label{fel_ellipsoids}
\end{equation}

% \addcontentsline{toc}{section}{References}
\bibliographystyle{unsrtnat}
\bibliography{references/biblio}
{\renewcommand{\bibname}{References}\bibliography{references/biblio}}

\end{document}

% \end{cbunit}