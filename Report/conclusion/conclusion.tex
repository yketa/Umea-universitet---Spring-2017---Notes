\documentclass[class=report, float=false, crop=false]{standalone}
%  \usepackage[subpreambles=true]{standalone}

\usepackage{pgf, tikz}
\usetikzlibrary{shapes.misc}
\usetikzlibrary{decorations.pathreplacing}

\tikzset{cross/.style={cross out, draw=black, minimum size=2*(#1-\pgflinewidth), inner sep=0pt, outer sep=0pt},
%default radius will be 1pt. 
cross/.default={0.25pt},
    point/.style={
    thick,
    draw=black,
    cross out,
    inner sep=0pt,
    minimum width=4pt,
    minimum height=4pt,
    },
}

\graphicspath{{introduction}}

% \begin{cbunit}

\begin{document}

\chapter*{Conclusion}
\label{conclusion}
\addcontentsline{toc}{chapter}{Conclusion}

In this work, we have presented models and methods which allowed us to investigate the jamming transition of spheroidal particles through shearing simulations.\\

Preliminary results are in good accordance with the existing literature, which reinforces the trust in our models, and also showed new and interesting results. We found that the rheological transition always happened at the same packing fractions for prolate spheroids, while it was strongly dependent of the aspect ratio for oblate spheroids (see part \ref{preliminary_rheological_transition}). In addition to this, we found that spheroidal particles in a sheared packing tended to angle themselves in a particular direction and with an important correlation for a packing fraction which does not correspond to the rheological transition or the jamming transition (see part \ref{results_orientation}). This packing fraction was also independent of the aspect ratio for prolate spheroids and not for oblate spheroids.\\

These interesting results could give new insights into the jamming transition. Indeed, we have seen in part \ref{rheology} that the rheology of the system and the orientations of the particles are intimately related to the way particles pack. Our results could then suggest that different mechanisms govern the jamming transition for prolate and oblate spheroids. Further studies are necessary to understand what are the causes and consequences of these phenomena.\\

An other relevant variable to describe the jamming transition is the mean number of contacts \cite{PRE68.011306,PRL88.075507,PRE91.062209,donev2007underconstrained,donev2004improving}. In particular, there have been recent claims that packings of ellipsoids would be hypostatic at jamming -- \textit{i.e.}, the total number of contacts is lesser than the total number of degrees of freedom -- but it was only demonstrated for static packings. Further studies with our methods could bring an other explanation to this phenomenon.\\

Finally, the study of the jamming transition as a critical phenomenon raises the question of the universality of our critical exponents. We showed that for prolate particles, the critical exponent associated to the vanishing of the pressure was close to 4 for prolate spheroids while it was close to 2.5 for spheres. Further studies are necessary to determine if the value of this exponent is universal for all spheroids, and if not how it should vary and why.

% \addcontentsline{toc}{section}{References}
\bibliographystyle{unsrtnat}
\bibliography{references/biblio}
{\renewcommand{\bibname}{References}\bibliography{references/biblio}}

\end{document}

% \end{cbunit}